\documentclass[margin,line]{res}

\oddsidemargin -.5in
\evensidemargin -.5in
\voffset -25pt
%\topmargin -.2in % start text higher on the page
\headsep 25pt
\textwidth=6.0in
\textheight=9.3in
\itemsep=0in
\parsep=0in

\usepackage{hyperref} % for links

% Headings
\pagestyle{myheadings}
\markright{Brian Koopman --- Curriculum Vitae}

\begin{document}  

\newcommand{\myname}{Brian Koopman}
\newlength{\mynamewidth}
\settowidth{\mynamewidth}{\namefont\myname}

\name{\hspace*{0.5\textwidth}\hspace{-0.5\mynamewidth} \myname \vspace*{.1in}}
% On the first page, have no header.
\thispagestyle{empty}

\begin{resume}                        

\section{\sc Contact Information}
Cornell University Physics Department \hfill (607) 255-0833\\
Laboratory of Elementary Particle Physics \hfill \href{mailto:bjk98@cornell.edu}{bjk98@cornell.edu}\\
324 Physical Sciences Building \hfill \href{http://www.briankoopman.com}{www.briankoopman.com}\\
Ithaca, NY 14850 USA
 
\section{\sc Education}
\textbf{Cornell University - Ithaca, NY}\\
    Ph.D., Physics (Advisor: Prof. Michael Niemack) \hfill \textbf{In Progress}\\
    M.S., Physics \hfill \textbf{June 2015}

\textbf{Clark University - Worcester, MA}\\
    B.A., \textit{summa cum laude}, with highest honors in Physics and Math \hfill \textbf{May 2012}\\
    GPA: 3.96 on a 4.00 scale

\section{\sc Research Interests}
Study of the of cosmic microwave background (CMB). Current work includes
hardware development for future upgrades to the Atacama Cosmology Telescope
Polarimeter, ACTPol, a CMB telescope located in the Atacama Desert in Chile.
This work includes optics design, detector assembly and measurement,
nanofabrication and general data analysis.

\section{\sc Honors and Awards}  % Old descriptions in comments
\textbf{NASA Space Technology Research Fellow}, NASA \hfill \textbf{2013 -- Present}\\
\\
\textbf{Dean's List - First Academic Honors}, Clark University \hfill \textbf{2008 -- 2012}\\ 
\\
\textbf{Roy S. Andersen '43 Award}, Clark University \hfill \textbf{2009}\\ 
% Award is given to one student for outstanding performance in PHYS120 and PHYS121
\\
\textbf{Albert C. Erickson '30 Summer Research Award}, Clark University \hfill \textbf{2009 -- 2010}\\ 
% Award is given to support undergraduate student summer research
\\
\textbf{Erickson Award for the Academic Year}, Clark University \hfill \textbf{2010 -- 2011}\\ 
% Award is given to one student for outstanding performance during an academic year

\section{\sc Research Experience}      
    \textbf{Graduate Researcher}, Cornell University, Ithaca, NY 14850 \hfill \textbf{December 2012 -- Present}
    \vspace*{1mm}
    \begin{itemize}
        \item [ ] Conducting research in observation cosmology, specifically working with
        the Atacama Cosmology Telescope Polarimeter as I work towards my thesis.
    \end{itemize} 

    \textbf{Student Researcher}, Clark University, Worcester, MA 01601 \hfill \textbf{Sept. 2011 -- May 2012}
    \vspace*{1mm}
    \begin{itemize}
        \item [ ] Conducted research in experimental condensed matter, specifically
        Scanning Tunneling Microscopy (STM), under the guidance of Prof. Michael Boyer.
        Aided in construction of the new STM laboratory. Performed analysis of Fe doped
        Bi$_2$Sr$_2$CaCu$_2$O$_{8+x}$ (Bi-2212) using custom software written for IDL.
    \end{itemize} 

    \textbf{Caltech REU Student}, LIGO Livingston, Livingston, LA 70754 \hfill \textbf{Summer 2011}
    \vspace*{1mm}
    \begin{itemize}
        \item [ ] Researched piezoelectric actuators for use in the
        Output Mode Cleaner (OMC) of the Laser Interferometer Gravitational Wave
        Observatory (LIGO) under the guidance of Dr. Valera Frolov. Participated in
        optical path construction and alignment of experimental OMC with non-linear
        planar ring oscillator (NPRO), Nd:YAG, laser. Collection of data with LIGO data
        acquisition system and processing with MATLAB.
    \end{itemize} 

    \textbf{Student Researcher}, Clark University, Worcester, MA 01601 \hfill \textbf{Summer 2009, 2010}
    \vspace*{1mm}
    \begin{itemize}
        \item [ ] Researched 1D antiferromagnetic chains
        Cu$_{(1-x)}$Zn$_{(x)}$(3,5-diClpy)$_2$Cl$_2$ and\\
        Cu(Py)$_2$Cl$_{2(1-x)}$Br$_{2(x)}$ under the guidance of Prof. Christopher
        Landee. Performed synthesis, simulation with the Algorithms and
        Libraries for Physics Simulations (ALPS), collected data using a SQUID
        Magnetometer and analyzed data using Origin 7.0.
    \end{itemize} 
    
\section{\sc Teaching Experience} 
    \textbf{Teaching Assistant}, Cornell University, Ithaca, NY 14850
    \vspace*{1mm}
    \begin{itemize} 
        \item [ ] PHYS2214 - Physics III: Oscillations, Waves, and Quantum Physics\hfill \textbf{Fall 2012}
        \item [ ] PHYS2213 - Physics II: Electromagnetism \hfill \textbf{Spring 2013}
    \end{itemize} 
    
    \textbf{Teaching Assistant}, Clark University, Worcester, MA 01601
    \vspace*{1mm}
    \begin{itemize} 
        \item [ ] PHYS127 - Computer Simulations \hfill \textbf{Spring 2011}
        \item [ ] MATH217 - Probability and Statistics \hfill \textbf{Fall 2010, 2011}
        \item [ ] CS120 - Introduction to Computing \hfill \textbf{Fall 2009, 2010}
    \end{itemize} 

\section{\sc Publications} 
\begin{enumerate}

\item[{2.}] ACTPol Collaboration (2015)
    \textit{The Atacama Cosmology Telescope: Lensing of CMB Temperature and Polarization Derived from Cosmic Infrared Background Cross-Correlation}, ApJ, 808, 7 (July 20, 2015); 
    \href{http://dx.doi.org/10.1088/0004-637X/808/1/7}{doi:10.1088/0004-637X/808/1/7}, 
    \href{http://arxiv.org/abs/1412.0626}{arXiv:1412.0626}.
\item[{1.}] ACTPol Collaboration (2014)
    \textit{The Atacama Cosmology Telescope: CMB Polarization at $200<\ell<9000$}, Journal of Cosmology and Astroparticle Physics 10(2014)007, (October 3, 2014); 
    \href{http://dx.doi.org/10.1088/1475-7516/2014/10/007}{doi:10.1088/1475-7516/2014/10/007}, 
    \href{http://arxiv.org/abs/1405.5524}{arXiv:1405.5524}.
\end{enumerate}

\section{\sc Conference Proceedings} 
\begin{enumerate}

\item[{3.}] Stacey, G.J., Parshley, S., Nikola, T., et. al.
    \textit{SWCam: the short wavelength camera for the CCAT Observatory},
    Proc. SPIE 9153, Millimeter, Submillimeter, and Far-Infrared Detectors and Instrumentation for Astronomy VII, 
    915310L (August 19, 2014); \href{http://dx.doi.org/10.1117/12.2057101}{doi:10.1117/12.2057101}
\item[{2.}] Grace, E., Beall, J., Bond, J.R., et. al.
    \textit{ACTPol: on-sky performance and characterization},
    Proc. SPIE 9153, Millimeter, Submillimeter, and Far-Infrared Detectors and Instrumentation for Astronomy VII, 
    915310 (July 23, 2014); \href{http://dx.doi.org/10.1117/12.2057243}{doi:10.1117/12.2057243}
\item[{1.}] Wheeler, J.D., Koopman, B., Gallardo, P., et. al.
    \textit{Antireflection coatings for submillimeter silicon lenses},
    Proc. SPIE 9153, Millimeter, Submillimeter, and Far-Infrared Detectors and Instrumentation for Astronomy VII, 
    91532Z (July 23, 2014); \href{http://dx.doi.org/10.1117/12.2057011}{doi:10.1117/12.2057011}
\end{enumerate}

\section{\sc Contributed Talks} 
\begin{enumerate}
    \item[{5.}] {\it Atacama Cosmology Telescope: Polarization calibration analysis for CMB measurements with ACTPol and Advanced ACTPol},
      APS April Meeting 2015, April 2015
    \item[{4.}] {\it Deep reactive ion etching of silicon anti-reflection coatings for sub-millimeter optics},
      SPIE Astronomical Telescopes and Instrumentation 2014, June 2014 (poster)
    \item[{3.}] {\it ACTPol: Status and preliminary CMB polarization results from the Atacama Cosmology Telescope},
      APS April Meeting 2014, April 2014
    \item[{2.}] {\it Development of Optics and Detectors for Advanced CMB Polarization Measurements},
      Cornell Graduate Student Seminar, November 2013
    \item[{1.}] {\it Scanning Tunneling Microsocpy of Fe Doped Bi$_2$Sr$_2$CaCu$_2$O$_{8+x}$},
      APS March Meeting 2012, February 2012
\end{enumerate}

\section{\sc Outreach and Service} 
Cornell Physics Graduate Society (PGS) Communications Officer \hfill \textbf{2013 -- 2014}\\
Organized PGS Outreach at Dragon Boat Festival \hfill \textbf{Summer 2013}\\
Organized PGS Outreach at Ithaca Festival \hfill \textbf{Summer 2013}\\
Expanding Your Horizons Conference Volunteer \hfill \textbf{2013}\\

\end{resume} 
\end{document}
